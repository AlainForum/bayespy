% Copyright 2012 by Jaakko Luttinen
%
% This file may be distributed and/or modified
%
% 1. under the LaTeX Project Public License and/or
% 2. under the GNU Public License.
%
% See the file doc/generic/pgf/licenses/LICENSE for more details.

% \ProvidesFileRCS[v\pgfversion] $Header: /cvsroot/pgf/pgf/generic/pgf/frontendlayer/tikz/libraries/tikzlibrarybayesnet.code.tex,v 0.1 2012/09/14 10:30:36 tantau Exp $

% Load other libraries
\usetikzlibrary{shapes,decorations,shadows}
\usetikzlibrary{decorations.pathmorphing}
\usetikzlibrary{decorations.shapes}
\usetikzlibrary{fadings}
\usetikzlibrary{patterns}
\usetikzlibrary{calc}
\usetikzlibrary{decorations.text}
\usetikzlibrary{decorations.footprints}
\usetikzlibrary{decorations.fractals}
\usetikzlibrary{shapes.gates.logic.IEC}
\usetikzlibrary{shapes.gates.logic.US}
\usetikzlibrary{fit,chains}
\usetikzlibrary{positioning}
\usepgflibrary{shapes}
\usetikzlibrary{scopes}
\usetikzlibrary{arrows}
\usetikzlibrary{backgrounds}

% Latent node
\tikzstyle{latent} = [circle,fill=white,draw=black,inner sep=1pt,
minimum size=20pt, font=\fontsize{10}{10}\selectfont, node distance=1]
% Observed node
\tikzstyle{obs} = [latent,fill=gray!25]
% Constant node
\tikzstyle{const} = [rectangle, inner sep=0pt, node distance=1]
% Factor node
\tikzstyle{factor} = [rectangle, fill=black,minimum size=5pt, inner
sep=0pt, node distance=0.4, font=\small]
% Deterministic node
\tikzstyle{det} = [latent, diamond]

% Plate node
\tikzstyle{plate} = [draw, rectangle, rounded corners, fit=#1]
% Invisible wrapper node
\tikzstyle{wrap} = [inner sep=0pt, fit=#1]
% Gate
\tikzstyle{gate} = [draw, rectangle, dashed, fit=#1]

% Caption node
\tikzstyle{caption} = [font=\footnotesize, node distance=0] %
\tikzstyle{plate caption} = [caption, node distance=0, inner sep=0pt,
below left=5pt and 0pt of #1.south east] %
\tikzstyle{factor caption} = [caption] %
\tikzstyle{every label} += [caption] %

\tikzset{>={triangle 45}}

%\pgfdeclarelayer{b}
%\pgfdeclarelayer{f}
%\pgfsetlayers{b,main,f}

% \factoredge [options] {inputs} {factors} {outputs}
\newcommand{\factoredge}[4][]{ %
  % Connect all nodes #2 to all nodes #4 via all factors #3.
  \foreach \f in {#3} { %
    \foreach \x in {#2} { %
      \draw[-,#1] (\x) edge[-] (\f) ;%
    } ;
    \foreach \y in {#4} { %
      \draw[->,#1] (\f) -- (\y) ;%
    } ;
  } ;
  %\draw[<-,#1] (#2) -- (#2-factor) edge[-] (#3) ;
}

% \edge [options] {inputs} {outputs}
\newcommand{\edge}[3][]{ %
  % Connect all nodes #2 to all nodes #3.
  \foreach \x in {#2} { %
    \foreach \y in {#3} { %
      \draw[->,#1] (\x) -- (\y) ;%
    } ;
  } ;
}

% \factor [options] {name} {caption}
\newcommand{\factor}[3][]{ %
  \node[factor, label={[name=#2-caption]#3}, #1] (#2) {} ;  %
}

% \plate [options] {name} {fitlist} {caption}
\newcommand{\plate}[4][]{ %
  \node[wrap=#3] (#2-wrap) {}; %
  \node[plate caption=#2-wrap] (#2-caption) {#4}; %
  \node[plate=(#2-wrap)(#2-caption), #1] (#2) {}; %
}

% \gate [options] {name} {fitlist} {inputs}
\newcommand{\gate}[4][]{ %
  \node[gate=#3, name=#2, #1, alias=#2-alias] {}; %
  \foreach \x in {#4} { %
    \draw [-*,thick] (\x) -- (#2-alias); %
  } ;%
}

% \vgate {name} {fitlist-left} {caption-left} {fitlist-right}
% {caption-right} {inputs}
\newcommand{\vgate}[6]{ %
  % Wrap the left and right parts
  \node[wrap=#2] (#1-left) {}; %
  \node[wrap=#4] (#1-right) {}; %
  % Draw the gate
  \node[gate=(#1-left)(#1-right)] (#1) {}; %
  % Add captions
  \node[caption, below left=of #1.north ] (#1-left-caption)
  {#3}; %
  \node[caption, below right=of #1.north ] (#1-right-caption)
  {#5}; %
  % Draw middle separation
  \draw [-, dashed] (#1.north) -- (#1.south); %
  % Draw inputs
  \foreach \x in {#6} { %
    \draw [-*,thick] (\x) -- (#1); %
  } ;%
}

% \hgate {name} {fitlist-top} {caption-top} {fitlist-bottom}
% {caption-bottom} {inputs}
\newcommand{\hgate}[6]{ %
  % Wrap the left and right parts
  \node[wrap=#2] (#1-top) {}; %
  \node[wrap=#4] (#1-bottom) {}; %
  % Draw the gate
  \node[gate=(#1-top)(#1-bottom)] (#1) {}; %
  % Add captions
  \node[caption, above right=of #1.west ] (#1-top-caption)
  {#3}; %
  \node[caption, below right=of #1.west ] (#1-bottom-caption)
  {#5}; %
  % Draw middle separation
  \draw [-, dashed] (#1.west) -- (#1.east); %
  % Draw inputs
  \foreach \x in {#6} { %
    \draw [-*,thick] (\x) -- (#1); %
  } ;%
}

\endinput

% % shapename, fitlist, caption, pos
% \newcommand{\plate}[4]{ %
%   \begin{pgfonlayer}{b}
%     \node (invis#1) [draw, color=white, inner sep=1pt, rectangle,
%     fit=#2] {};
%   \end{pgfonlayer}
%   \begin{pgfonlayer}{f}
%     \node (capt#1) [ below left=0 pt of invis#1.south east,
%     xshift=2pt,yshift=1pt] {\footnotesize{#3}}; %
%     \node (#1) [draw,inner sep=2pt, rectangle,fit=(invis#1)
%     (capt#1),#4] {}; %
%   \end{pgfonlayer}
% }


% \newcommand{\shiftedplate}[5]{ %
%   \begin{pgfonlayer}{b}
%     \node (invis#1) [draw, color=white, inner sep=0pt, rectangle,
%     fit=#2] {};
%   \end{pgfonlayer}
%   \begin{pgfonlayer}{f}
%     \node (capt#1) [#5, xshift=2pt] {\footnotesize{#3}}; %
%     \node (#1) [draw,inner sep=2pt, rectangle, fit=(invis#1) (capt#1),
%     #4] {}; %
% \end{pgfonlayer}
% }

% %shapename, pos, caption, in1, in2, out, captpos
% \newcommand{\twofactor}[7]{ %
%   \node (#1) [factor] at #2 {};%
%   \node (capt#1) [#7 of #1]{\footnotesize{#3}}; %
%   \draw [-] (#4) -- (#1) ; %
%   \draw [-] (#5) -- (#1) ; %
%   \draw [->,thick] (#1) -- (#6); %
% }

% %shapename, pos, caption, in, out, captpos
% \newcommand{\factorexperimental}[6]{ %
%   \node [factor] (#1) at #2 {}; %
%   \node (#1-caption) [#6] at #2 {\footnotesize{#3}}; %
%   \draw [-] (#4) -- (#1) ; %
%   \draw [->,thick] (#1) -- (#5); %
% }

% % shapename, pos, caption, in, out, captpos
% \newcommand{\factor}[6]{ %
%   \node (#1) [factor] at #2 {}; %
%   \node (#1-caption) [#6 of #1]{\footnotesize{#3}}; %
%   \draw [-] (#4) -- (#1) ; %
%   \draw [->,thick] (#1) -- (#5); %
% }

% % name, --, caption, pos
% \newcommand{\nofactor}[4]{ %
%   \node (#1) [factor, #2] {}; %
%   \node (capt#1) [#4 of #1]{\footnotesize{#3}}; %
% }

% %shapename,  fitlist, caption
% \newcommand{\namedgate}[3]{ %
%   \begin{pgfonlayer}{b}
%     \node (invisgate#1) [rectangle, draw, color=white, fit=#2] {};
%   \end{pgfonlayer}
%   \node (gatecapt#1) [ above right=0 pt of invisgate#1.north west,
%   xshift=-1pt ] {\footnotesize{#3}}; %
%   \node (#1) [rectangle,draw,dashed, inner sep=2pt,
%   fit=(invisgate#1)(gatecapt#1)]{}; %
% }

% %shapename,  fitlist, caption
% \newcommand{\gate}[3]{ %
%   \node (#1) [rectangle,draw,dashed, inner sep=2pt, fit=#2]{};
% }

% %shapename,  fitlist1, fitlist2, caption1, caption2
% \newcommand{\vertgate}[5]{ %
%   \begin{pgfonlayer}{b}
%     \node (invisgateleft#1) [rectangle, draw, color=white, fit=#2]
%     {}; %
%     \node (invisgateright#1) [rectangle, draw, color=white, fit=#4]
%     {}; %
%   \end{pgfonlayer}
%   \node (gatecaptleft#1) [ above left=0 pt of invisgateleft#1.north
%   east, xshift=1pt ]{\footnotesize{#3}} ; %
%   \node (gatecaptright#1) [ above right=0 pt of invisgateright#1.north
%   west, xshift=-1pt ] {\footnotesize{#5}}; %
%   \node (#1) [rectangle,draw,dashed, inner sep=2pt,
%   fit=(invisgateleft#1)(gatecaptleft#1)(invisgateright#1)(gatecaptright#1)]{}; %
%   \draw [-, dashed] (#1.north) -- (#1.south); %
% }


% \newcommand{\vertgateSpec}[5]{
%   \begin{pgfonlayer}{b}
%     \node (invisgateleft#1) [rectangle, draw, color=white,  fit=#2] {}; %
%     \node (invisgateright#1) [rectangle, draw, color=white,  fit=#4] {}; %
%   \end{pgfonlayer}
%   \node (gatecaptleft#1) [ above left=0 pt of invisgateleft#1.north
%   east, xshift=1pt ]{\footnotesize{#3}}; %
%   \node (gatecaptright#1) [ above right=0 pt of invisgateright#1.north
%   west, xshift=-1pt ] {\footnotesize{#5}}; %
%   \node (#1) [rectangle,draw,dashed, inner sep=2pt,
%   fit=(invisgateleft#1)(gatecaptleft#1)(invisgateright#1)(gatecaptright#1)]{}; %
%   \draw [-, dashed] (#1.70) -- (#1.290); %
% }

% \newcommand{\horgate}[5]{
%   \begin{pgfonlayer}{b}
%     \node (invisgateleft#1) [rectangle, draw, color=white, fit=#2]
%     {}; %
%     \node (invisgateright#1) [rectangle, draw, color=white, fit=#4]
%     {}; %
%   \end{pgfonlayer}
%   \node (gatecaptleft#1) [ above right=0 pt of invisgateleft#1.south
%   west, xshift=1pt ]{\footnotesize{#3}}; %
%   \node (gatecaptright#1) [ below right=0 pt of invisgateright#1.north
%   west, xshift=-1pt ] {\footnotesize{#5}}; %
%   \node (#1) [rectangle,draw,dashed, inner sep=2pt,
%   fit=(invisgateleft#1)(gatecaptleft#1)(invisgateright#1)(gatecaptright#1)]{}; %
%   \draw [-, dashed] (#1.west) -- (#1.east); %
% }

% \newcommand{\horogate}[5]{
%   \begin{pgfonlayer}{b}
%     \node (invisgateleft#1) [rectangle, draw, color=white, fit=#2]
%     {}; %
%     \node (invisgateright#1) [rectangle, draw, color=white, fit=#4]
%     {}; %
%   \end{pgfonlayer}
%   \node (#1) [rectangle,draw,dashed, inner sep=2pt,
%   fit=(invisgateleft#1)(invisgateright#1)]{}; %
%   \node (gatecaptleft#1) [ above right=0 pt of #1.west, xshift=0pt
%   ]{\footnotesize{#3}}; %
%   \node (gatecaptright#1) [ below right=0 pt of #1.west, xshift=0pt ]
%   {\footnotesize{#5}}; %
%   \draw [-, dashed] (#1.west) -- (#1.east); %
% }


% \newcommand{\vertogate}[5]{
%   \begin{pgfonlayer}{b}
%     \node (invisgateleft#1) [rectangle, draw, color=white, fit=#2]
%     {}; %
%     \node (invisgateright#1) [rectangle, draw, color=white, fit=#4]
%     {}; %
%   \end{pgfonlayer}
%   \node (#1) [rectangle,draw,dashed, inner sep=2pt,
%   fit=(invisgateleft#1)(invisgateright#1)]{}; %
%   \node (gatecaptleft#1) [ below left=0 pt of #1.north, xshift=0pt
%   ]{\footnotesize{#3}}; %
%   \node (gatecaptright#1) [ below right=0 pt of #1.north, xshift=0pt ]
%   {\footnotesize{#5}}; %
%   \draw [-, dashed] (#1.north) -- (#1.south); %
% }

% \endinput
